% ============================================================
%  Sezione 5 — Discussione e Conclusioni
% ============================================================
\section{Discussione e Conclusioni}
\label{sec:discussione}

% ------------------------------------------------------------
\subsection{Interpretazione dei Risultati}
\label{sec:disc:interpretazione}

I risultati degli esperimenti convergono verso un'indicazione chiara: la
qualità del risultato finale dipende in modo cruciale dalla catena completa
del pipeline, e la scelta del metodo di stima del moto ha un peso almeno
pari a quello del metodo di smoothing. Questo è un punto non scontato: si
potrebbe essere tentati di pensare che uno smoother sufficientemente aggressivo
possa compensare le imprecisioni della stima del moto a monte, ma i dati
mostrano che così non è. Il fatto che la Media Mobile applicata a RAFT
raggiunga uno stability score di $95.1$ mentre la stessa Media Mobile
applicata a SIFT si fermi a $82.3$ — pur con parametri di smoothing identici
— dimostra che la qualità della traiettoria raw condiziona strettamente la
qualità di quella smoothata.

Il trade-off tra stability score e fidelity score merita un'attenzione
particolare. Come si è visto, i due indicatori tendono a muoversi in direzioni
opposte: ogni aumento dello stability score comporta un costo in termini di
fedeltà al moto originale, e viceversa. Questo non è un artefatto del sistema
implementato, ma riflette un limite fondamentale di qualsiasi approccio di
smoothing lineare: una finestra di integrazione larga elimina più rumore ma
degrada più segnale utile. Il Filtro di Kalman si posiziona lungo questa curva
di compromesso in modo più vantaggioso rispetto alla Media Mobile, poiché il
suo modello cinematico lo rende selettivo rispetto alla frequenza: attenua
le oscillazioni rapide mantenendo quelle lente, invece di trattare
uniformemente tutte le componenti spettrali come fa la convoluzione con un
kernel rettangolare.

Un risultato particolarmente istruttivo emerge dal confronto RAFT + Media
Mobile: questo scenario ottiene il massimo stability score ($95.1$) ma anche
il minimo fidelity score ($26.4$). La ragione va ricercata proprio nella
qualità delle stime RAFT. Poiché il flusso ottico denso è già quasi privo di
jitter ad alta frequenza, la Media Mobile non trova molto da eliminare su quel
fronte; compensa invece abbattendo indiscriminatamente tutta la variabilità
del segnale, inclusi i movimenti intenzionali. In pratica, uno smoother così
conservativo ``pialla'' una traiettoria già quasi uniforme, producendo un video
con telecamera quasi ferma che potrebbe risultare artificiale e innaturale allo
spettatore. Questo suggerisce che la scelta dei parametri di smoothing debba
essere adattata alla qualità dell'estimatore a monte: con RAFT è sufficiente
un raggio di Media Mobile molto più piccolo, o un Kalman con $R$ ridotto,
per ottenere una stabilizzazione efficace senza perdere il moto intenzionale.

% ------------------------------------------------------------
\subsection{Limiti del Sistema}
\label{sec:disc:limiti}

Il sistema presenta alcuni limiti che vale la pena riconoscere esplicitamente.
Un primo limite riguarda la latenza complessiva: più di $40$~secondi per
processare una sequenza sono accettabili in un contesto offline, ma
proibitivi per applicazioni real-time. RAFT beneficia già dell'accelerazione
GPU tramite CUDA, il che lo rende competitivo con SIFT su CPU nonostante la
maggiore complessità del modello; tuttavia, per avvicinarsi a prestazioni
real-time sarebbe necessario ridurre la risoluzione di elaborazione o adottare
varianti più leggere come RAFT Small, a scapito dell'accuratezza. SIFT, invece,
gira interamente su CPU e non può beneficiare di accelerazione GPU in modo
diretto, rappresentando il vero collo di bottiglia in scenari ad alta
risoluzione.

Sul piano della robustezza, SIFT mostra una vulnerabilità nota in scene con
poca texture o in presenza di motion blur intenso: in questi contesti il numero
di keypoint rilevabili può scendere sotto la soglia minima per RANSAC, e la
funzione ritorna $(0, 0, 0)$ per l'intero frame, introducendo un'imprecisione
nella traiettoria cumulativa che si propaga ai frame successivi. RAFT è
strutturalmente immune a questo problema poiché opera su tutti i pixel, ma
è soggetto a propri artefatti nelle scene con superfici riflettenti o in
condizioni di sottoesposizione severa.

Un altro limite riguarda il modello di moto adottato: il sistema stima e
corregge esclusivamente la trasformazione di similarità nel piano dell'immagine
(traslazione + rotazione + scala uniforme), ignorando le componenti di moto
tridimensionale — variazioni di prospettiva, rollio fuori piano, variazioni
focali. Questo modello è adeguato per le tipiche oscillazioni da camera a mano,
ma potrebbe rivelarsi insufficiente per sequenze acquisite con droni o con
camere montate su veicoli in rapido movimento.

Infine, il fidelity score basato su $R^2$ fornisce una misurazione utile ma
non cattura la qualità percepita dall'osservatore umano. Due traiettorie con
lo stesso fidelity score possono produrre risultati visivamente molto diversi
a seconda di dove si concentrano i residui di errore: un errore concentrato
in pochi frame è più disturbante dello stesso errore distribuito uniformemente
lungo l'intera sequenza.

% ------------------------------------------------------------
\subsection{Linee di Sviluppo Future}
\label{sec:disc:futuro}

Diverse direzioni di sviluppo sembrano promettenti. Sul fronte delle
prestazioni, un guadagno immediato sarebbe ottenibile elaborando i frame
in batch anziché uno alla volta, riducendo l'overhead dei trasferimenti
tra RAM e VRAM per RAFT e sfruttando meglio il parallelismo della GPU.
In parallelo, l'adozione di un modello di moto omografico — in grado di
stimare la trasformazione prospettica completa tra due frame — permetterebbe
di gestire correttamente le rotazioni tridimensionali e le variazioni di
campo visivo.

Sul fronte dello smoothing, un'esplorazione interessante riguarderebbe
l'adozione di filtri adattativi che modulino automaticamente i parametri
$Q$ e $R$ in funzione del contenuto dinamico della scena: in una sequenza
con inquadratura statica è ragionevole imporre un smoothing più aggressivo
rispetto a una panoramica, e un sistema di rilevamento automatico del tipo
di moto potrebbe guidare questa adattività.

Un'altra direzione è l'integrazione di metriche di qualità percettiva — come
SSIM o VMAF — per affiancare le metriche geometriche attualmente implementate,
avvicinando la valutazione al giudizio soggettivo dell'utente finale. Questo
consentirebbe anche di costruire funzioni di ottimizzazione più allineate con
la reale esperienza visiva, aprendo la strada a tecniche di tuning automatico
dei parametri.

% ------------------------------------------------------------
\subsection{Conclusioni}
\label{sec:disc:conclusioni}

Questo progetto ha realizzato e valutato una pipeline completa di
stabilizzazione video che integra due approcci alla stima del moto — il
classico SIFT e il modello deep learning RAFT — con due strategie di smoothing
della traiettoria, la Media Mobile e il Filtro di Kalman, il tutto accessibile
tramite una dashboard interattiva che permette di esplorarne i parametri in
tempo reale.

I risultati sperimentali mostrano che RAFT produce stime del moto
sistematicamente più accurate di SIFT, con un RMS raw inferiore di quasi tre
volte, e che questa qualità si traduce direttamente in stability score più
elevati a valle dello smoothing. Il Filtro di Kalman offre il miglior
bilanciamento tra riduzione del jitter e preservazione del moto intenzionale,
posizionandosi in modo più vantaggioso sul trade-off stabilità--fedeltà
rispetto alla Media Mobile indipendentemente dal metodo di stima. La
combinazione RAFT + Kalman è quella che risulta più equilibrata nei due
indicatori simultaneamente, con uno stability score di $59.2$ e un fidelity
score di $59.0$, mentre RAFT + Media Mobile è preferibile quando l'obiettivo
è la massima stabilità e si accetta una perdita di fedeltà al moto originale.

Dal punto di vista didattico, il progetto ha offerto l'opportunità di
approfondire concretamente tanto i fondamenti classici della visione artificiale
— scala di interesse, invarianza ai descrittori, stima robusta con RANSAC —
quanto le architetture deep learning per l'optical flow, verificandone i
vantaggi e i limiti in un'applicazione reale. La dashboard realizzata
costituisce uno strumento di esplorazione immediata e intuitiva di questi
fenomeni, rendendo visibili e quantificabili differenze che altrimenti
rimarrebbero difficili da apprezzare senza strumenti di analisi dedicati.
