% ============================================================
%  Sezione 1 — Introduzione
% ============================================================
\section{Introduzione}

\subsection{Motivazione}

La stabilizzazione video è uno dei problemi fondamentali nell'elaborazione digitale
di sequenze d'immagini. Quando un video viene acquisito da una telecamera tenuta a
mano, montata su un veicolo in movimento o soggetta a vibrazioni meccaniche, il
risultato è una sequenza affetta da \emph{jitter}: piccole oscillazioni ad alta
frequenza della posizione e dell'orientamento del sensore che rendono il filmato
visivamente fastidioso e, in applicazioni più sofisticate, difficile da analizzare
automaticamente.

Il problema è rilevante in numerosi contesti pratici: dalla videografia amatoriale
ai sistemi di sorveglianza, dalla guida autonoma alla robotica, fino all'analisi
sportiva e alle riprese d'azione. In tutti questi scenari è desiderabile separare
il \emph{moto intenzionale} della telecamera (panoramiche, zoom, inseguimento di
un soggetto) dall'instabilità indesiderata, attenuando quest'ultima senza
alterare il contenuto informativo del video.

\subsection{Obiettivi del Progetto}

Questo progetto ha due obiettivi principali:

\begin{enumerate}
    \item \textbf{Implementare una pipeline completa di stabilizzazione video}
          che comprenda la stima del movimento inter-frame, lo
          smoothing della traiettoria stimata e la conseguente correzione
          geometrica dei frame.

    \item \textbf{Confrontare sistematicamente} due famiglie di approcci alla
          stima del moto — metodi classici basati su feature (SIFT) e
          metodi deep learning (RAFT) — in combinazione con due strategie
          di smoothing della traiettoria: la Media Mobile e il Filtro di
          Kalman.
\end{enumerate}

Il confronto viene condotto in modo rigoroso attraverso un insieme di metriche
quantitative — RMS displacement, jitter reduction, stability score e fidelity
score — i cui risultati sono integrati in una dashboard interattiva realizzata
con Streamlit, che consente di esplorare in tempo reale l'effetto di ciascun
parametro.

\subsection{Approccio Metodologico}

La pipeline implementata segue uno schema a tre stadi:

\begin{enumerate}
    \item \textbf{Motion Estimation}: stima del moto globale
          $(dx_t,\, dy_t,\, d\theta_t)$ tra coppie di frame consecutivi,
          utilizzando alternativamente SIFT o RAFT.

    \item \textbf{Trajectory Smoothing}: integrazione delle stime incrementali
          in una traiettoria cumulativa $\mathbf{p}_t = \sum_{i=1}^{t}
          (dx_i, dy_i, d\theta_i)$, seguita da una fase di smoothing con
          Media Mobile o Filtro di Kalman per ottenere la traiettoria
          target $\hat{\mathbf{p}}_t$.

    \item \textbf{Frame Correction}: applicazione della trasformazione affine
          di correzione $\mathbf{c}_t = \hat{\mathbf{p}}_t - \mathbf{p}_t$
          a ogni frame, con zoom adattivo per eliminare i bordi neri
          introdotti dal warping.
\end{enumerate}

\subsection{Struttura della Relazione}

La relazione è organizzata come segue. La Sezione~\ref{sec:teoria} illustra
il background teorico dei metodi utilizzati: optical flow, SIFT, RAFT, Media
Mobile e Filtro di Kalman. La Sezione~\ref{sec:implementazione} descrive le
scelte implementative e l'architettura del sistema. La
Sezione~\ref{sec:esperimenti} presenta il setup sperimentale e i risultati
ottenuti. La Sezione~\ref{sec:discussione} discute i risultati e traccia le
conclusioni.
