% ============================================================
%  Sezione 4 — Esperimenti
% ============================================================
\section{Esperimenti}
\label{sec:esperimenti}

% ------------------------------------------------------------
\subsection{Setup Sperimentale}
\label{sec:exp:setup}

Tutti gli esperimenti descritti in questa sezione sono stati condotti sulla
medesima sequenza video, acquisita con una fotocamera tenuta a mano. La risoluzione
di elaborazione è stata fissata a $1280$ pixel di larghezza per entrambi gli
esperimenti, mantenendo il rapporto d'aspetto originale.

Le due configurazioni confrontate corrispondono ai due metodi di stima del
moto implementati: l'\textbf{Esperimento~1} utilizza SIFT con i parametri di
default (500 keypoint massimi, soglia di contrasto $0.04$, soglia sugli edge
$10$, $\sigma = 1.6$, ratio test $0.75$); l'\textbf{Esperimento~2} adotta
RAFT Large con 200 punti di campionamento sul flusso denso per la stima della
rotazione. Per rendere il confronto equo, i parametri di smoothing sono
identici nei due esperimenti: raggio della Media Mobile $r = 30$, e per il
Filtro di Kalman $Q = 0.0005$ e $R = 2.0$. Lo zoom è abilitato al $10\%$ in
entrambi i casi.

I due metodi operano su hardware diverso: SIFT gira interamente su CPU,
mentre RAFT sfrutta la GPU tramite CUDA. La fase di smoothing — avendo
complessi\-tà $O(N)$ — contribuisce in modo trascurabile al tempo totale
in entrambi i casi (circa $0.28$~s), concentrando l'interesse sulle
prestazioni della sola stima del moto.

% ------------------------------------------------------------
\subsection{Esperimento 1: SIFT Baseline}
\label{sec:exp:sift}

Nel primo esperimento la stima del moto ha richiesto complessivamente $40.7$~s
per l'intera sequenza, con un tempo totale di elaborazione di $41.2$~s.
La Tabella~\ref{tab:exp1} riporta le metriche principali per i due metodi
di smoothing applicati alla traiettoria stimata da SIFT.

\begin{table}[H]
\centering
\caption{Metriche — Esperimento 1 (SIFT, $r{=}30$, $Q{=}0.0005$, $R{=}2.0$).}
\label{tab:exp1}
\renewcommand{\arraystretch}{1.25}
\begin{tabular}{lcc}
\toprule
\textbf{Metrica} & \textbf{Media Mobile} & \textbf{Kalman} \\
\midrule
Stability Score $[0\text{--}100]$         & $82.3$  & $45.6$  \\
Fidelity Score $[0\text{--}100]$          & $41.7$  & $67.0$  \\
\midrule
RMS raw $x$ [px]                          & \multicolumn{2}{c}{$7.71$} \\
RMS smoothed $x$ [px]                     & $6.71$  & $7.34$  \\
RMS raw $y$ [px]                          & \multicolumn{2}{c}{$4.58$} \\
RMS smoothed $y$ [px]                     & $3.90$  & $4.24$  \\
RMS raw $\theta$ [°]                      & \multicolumn{2}{c}{$0.133$} \\
RMS smoothed $\theta$ [°]                 & $0.028$ & $0.100$ \\
\midrule
Jitter Reduction $x$ [\%]                 & $77.3$  & $43.4$  \\
Jitter Reduction $y$ [\%]                 & $79.1$  & $48.3$  \\
Jitter Reduction $\theta$ [\%]            & $98.6$  & $44.6$  \\
\midrule
Max offset $x$ [px]                       & $53.5$  & $38.6$  \\
Max offset $y$ [px]                       & $25.0$  & $16.2$  \\
\midrule
Tempo motion estimation [s]               & \multicolumn{2}{c}{$40.7$} \\
\bottomrule
\end{tabular}
\end{table}

La Media Mobile raggiunge uno stability score di $82.3$, grazie a una forte
riduzione del jitter su tutti e tre gli assi — particolarmente marcata sulla
rotazione ($98.6\%$). Tuttavia il fidelity score si ferma a $41.7$: la
convoluzione con la finestra uniforme attenua in modo indiscriminato ogni
oscillazione, incluse quelle a frequenza più bassa che possono corrispondere
a movimenti intenzionali della telecamera. Lo dimostrano anche i valori del
RMS smoothed: l'RMS sull'asse $x$ scende da $7.71$ a $6.71$~px, ma la
riduzione è quasi interamente attribuibile all'attenuazione del jitter piuttosto
che a una precisa inseguimento della traiettoria raw.

Il Filtro di Kalman presenta un profilo complementare. Lo stability score
scende a $45.6$ — la jitter reduction è più modesta, attorno al $44$--$48\%$
per $x$ e $y$ — ma il fidelity score sale a $67.0$, confermando che il filtro
riesce a distinguere meglio tra oscillazioni di breve durata e movimenti
intenzionali a più bassa frequenza. La componente di velocità nel vettore
di stato consente al Kalman di \emph{predire} l'evoluzione della traiettoria
invece di limitarsi a mediarla, producendo una stima più fedele al moto
originale. L'offset massimo applicato è sensibilmente minore rispetto alla
Media Mobile ($38.6$ vs $53.5$~px su $x$), a indicare che la traiettoria
smoothata non si discosta eccessivamente da quella grezza.

\begin{figure}[H]
    \centering
    \includegraphics[width=\linewidth]{SIFT_1}
    \caption{Esperimento 1 (SIFT): confronto tra traiettoria grezza, smoothata
    con Media Mobile e smoothata con Filtro di Kalman sugli assi $x$, $y$ e $\theta$.}
    \label{fig:sift1}
\end{figure}

% ------------------------------------------------------------
\subsection{Esperimento 2: RAFT Baseline}
\label{sec:exp:raft}

Il secondo esperimento sostituisce SIFT con RAFT Large mantenendo invariati
tutti gli altri parametri. La stima del moto ha richiesto $43.1$~s, circa
$2.4$~s in più rispetto a SIFT: la differenza è attribuibile al tempo di
inferenza della rete e alla gestione del tensore su CPU. La
Tabella~\ref{tab:exp2} riporta le metriche per questo esperimento.

\begin{table}[H]
\centering
\caption{Metriche — Esperimento 2 (RAFT Large, $r{=}30$, $Q{=}0.0005$, $R{=}2.0$).}
\label{tab:exp2}
\renewcommand{\arraystretch}{1.25}
\begin{tabular}{lcc}
\toprule
\textbf{Metrica} & \textbf{Media Mobile} & \textbf{Kalman} \\
\midrule
Stability Score $[0\text{--}100]$         & $95.1$  & $59.2$  \\
Fidelity Score $[0\text{--}100]$          & $26.4$  & $59.0$  \\
\midrule
RMS raw $x$ [px]                          & \multicolumn{2}{c}{$2.69$} \\
RMS smoothed $x$ [px]                     & $1.01$  & $1.81$  \\
RMS raw $y$ [px]                          & \multicolumn{2}{c}{$1.40$} \\
RMS smoothed $y$ [px]                     & $0.311$ & $0.830$ \\
RMS raw $\theta$ [°]                      & \multicolumn{2}{c}{$0.138$} \\
RMS smoothed $\theta$ [°]                 & $0.028$ & $0.102$ \\
\midrule
Jitter Reduction $x$ [\%]                 & $90.9$  & $57.9$  \\
Jitter Reduction $y$ [\%]                 & $98.2$  & $67.4$  \\
Jitter Reduction $\theta$ [\%]            & $97.4$  & $45.6$  \\
\midrule
Max offset $x$ [px]                       & $20.3$  & $14.6$  \\
Max offset $y$ [px]                       & $13.0$  & $8.15$  \\
\midrule
Tempo motion estimation [s]               & \multicolumn{2}{c}{$43.1$} \\
\bottomrule
\end{tabular}
\end{table}

Il primo elemento che emerge con evidenza è l'RMS raw di RAFT: $2.69$~px
sull'asse $x$ contro i $7.71$~px di SIFT. Un valore così basso non indica
necessariamente che il video originale sia più stabile, ma piuttosto che
RAFT produce stime del moto globale intrinsecamente meno rumorose. Il flusso
denso, calcolato sull'intera immagine con una rete addestrata su milioni di
esempi, è molto meno suscettibile ai falsi match che affliggono il pipeline
SIFT in presenza di ripetizioni di pattern, zone a bassa texture o motion
blur.

Con la Media Mobile, lo stability score raggiunge $95.1$ — il valore più alto
tra i quattro scenari testati — e la jitter reduction sull'asse $y$ tocca il
$98.2\%$. Il prezzo da pagare è un fidelity score di appena $26.4$: poiché
le stime RAFT sono già quasi prive di jitter, applicare una finestra media di
30 frame ha l'effetto di eliminare anche i movimenti intenzionali a frequenza
medio-bassa, producendo una traiettoria smoothata che si discosta
significativamente da quella raw. In altri termini, la Media Mobile è
sprecata su un segnale già pulito.

Il Kalman nel contesto RAFT mostra invece un profilo molto più bilanciato.
Lo stability score ($59.2$) supera il corrispondente SIFT ($45.6$), mentre il
fidelity score ($59.0$) è notevolmente più alto rispetto alla Media Mobile
RAFT ($26.4$) e si avvicina al valore ottenuto dal Kalman su SIFT ($67.0$).
Il Kalman riesce a sfruttare la qualità delle stime RAFT senza distruggerle:
il suo modello a velocità costante evita di interpretare come jitter quelle
componenti di bassa frequenza che invece corrispondono a movimenti reali della
telecamera.

\begin{figure}[H]
    \centering
    \includegraphics[width=\linewidth]{RAFT_1}
    \caption{Esperimento 2 (RAFT): confronto tra traiettoria grezza, smoothata
    con Media Mobile e smoothata con Filtro di Kalman sugli assi $x$, $y$ e $\theta$.}
    \label{fig:raft1}
\end{figure}

% ------------------------------------------------------------
\subsection{Confronto Diretto tra i Quattro Scenari}
\label{sec:exp:confronto}

La Tabella~\ref{tab:confronto} sintetizza i risultati dei quattro scenari
(metodo di stima $\times$ metodo di smoothing) nelle due metriche di
valutazione più rappresentative.

\begin{table}[H]
\centering
\caption{Confronto complessivo: Stability Score e Fidelity Score.}
\label{tab:confronto}
\renewcommand{\arraystretch}{1.3}
\begin{tabular}{llcc}
\toprule
\textbf{Motion Est.} & \textbf{Smoothing} & \textbf{Stability} & \textbf{Fidelity} \\
\midrule
SIFT  & Media Mobile & $82.3$ & $41.7$ \\
SIFT  & Kalman       & $45.6$ & $67.0$ \\
RAFT  & Media Mobile & $\mathbf{95.1}$ & $26.4$ \\
RAFT  & Kalman       & $59.2$ & $\mathbf{59.0}$ \\
\bottomrule
\end{tabular}
\end{table}

La tabella mette in luce due tendenze sistematiche. Prima: a parità di metodo
di smoothing, RAFT produce sempre uno stability score più alto di SIFT
(+12.8 punti con Media Mobile, +13.6 con Kalman), segno che la qualità della
stima del moto ha un impatto diretto sulla qualità del risultato finale.
Seconda: a parità di metodo di stima, la Media Mobile supera sempre il Kalman
in stabilità ma ne è sempre inferiore in fedeltà; lo scarto in stabilità è
di $36.7$ punti per SIFT e $35.9$ per RAFT, mentre lo scarto in fedeltà è di
$25.3$ e $32.6$ punti rispettivamente, evidenziando che il compromesso
stabilità--fedeltà è una caratteristica strutturale dei due smoothing e non
dipende dal metodo di stima a monte.

Vale la pena notare il comportamento asimmetrico del fidelity score: con
SIFT la Media Mobile ottiene $41.7$ contro $67.0$ del Kalman, uno scarto di
$25.3$ punti. Con RAFT, lo stesso confronto produce uno scarto di $32.6$
punti ($26.4$ vs $59.0$). Questo suggerisce che quanto più le stime raw sono
accurate (cioè quanto meno rumorose), tanto più la Media Mobile è
\emph{deleteria} per la fedeltà al moto intenzionale: avendo poco jitter da
eliminare, la sua azione di smoothing si abbatte quasi interamente sul
contenuto informativo del segnale.

% ------------------------------------------------------------
\subsection{Costo Computazionale}
\label{sec:exp:tempo}

Come atteso, la fase di smoothing è computazionalmente trascurabile in
entrambi gli esperimenti: $0.28$~s per la Media Mobile e $0.28$--$0.30$~s
per il Kalman, indipendentemente dal metodo di stima a monte. Il carico
computazionale è dominato quasi interamente dalla stima del moto: $40.7$~s
per SIFT (su CPU) e $43.1$~s per RAFT (su GPU con CUDA).

Il fatto che RAFT su GPU richieda un tempo paragonabile a SIFT su CPU è
rilevante: nonostante la rete abbia milioni di parametri, il singleton del
modello viene caricato una sola volta e l'inferenza viene eseguita in
modalità \texttt{no\_grad}, minimizzando l'overhead di PyTorch. Il collo
di bottiglia per RAFT è il trasferimento dei tensori tra RAM e VRAM e la
latenza dei kernel CUDA per frame di risoluzione relativamente piccola
($1280$ px): su sequenze ad alta risoluzione o con GPU più recenti il
vantaggio rispetto a SIFT su CPU sarebbe più marcato. SIFT, dal canto suo,
benefilcerebbe relativamente poco da una GPU, essendo un algoritmo basato
su operazioni sparse difficilmente parallelizzabili.
